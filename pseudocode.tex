%%%%%%%%%%%%%%%%%%%%%%%%%%%%%%%%%%%%%%%%%%%%%%%%%%%%%%%%%%%%%%%%%%%%%%%%%%%%%%%%
% pseudocode.tex
% An example demonstrating the use of the algorithmicx package to write
% pseudocode.
% http://mirror.csclub.uwaterloo.ca/CTAN/macros/latex/contrib/algorithmicx/algorithmicx.pdf
% https://github.com/mhyee/latex-examples/
%%%%%%%%%%%%%%%%%%%%%%%%%%%%%%%%%%%%%%%%%%%%%%%%%%%%%%%%%%%%%%%%%%%%%%%%%%%%%%%%


% LaTeX Preamble
% Load packages and set options as needed
%%%%%%%%%%%%%%%%%%%%%%%%%%%%%%%%%%%%%%%%%%%%%%%%%%%%%%%%%%%%%%%%%%%%%%%%%%%%%%%%

% Set the document class to "article"
% Pass it "letterpaper" option
\documentclass[letterpaper]{article}

% We don't need the special font encodings, but still
% good practice to include these. See:
%
% http://tex.stackexchange.com/questions/664/why-should-i-use-usepackaget1fontenc
% http://dsanta.users.ch/resources/type1.html
\usepackage[T1]{fontenc}
\usepackage{ae,aecompl}

% For mathematical symbols such as floors in our pseudocode
\usepackage{amsmath}

% Use the algorithmicx package for pseudocode
\usepackage{algorithm}
\usepackage{algpseudocode}

% Disable page numbering
\pagestyle{empty}

% For fancy borders around figures. This is optional.
\restylefloat{figure}


% Begin the actual typesetting, by starting the "document" environment
%%%%%%%%%%%%%%%%%%%%%%%%%%%%%%%%%%%%%%%%%%%%%%%%%%%%%%%%%%%%%%%%%%%%%%%%%%%%%%%%
\begin{document}

% A numbered list
\begin{enumerate}

  \item
  % Note that `pdflatex` will need to be invoked at least twice for figure
  % reference numbers to be resolved.
  Exercise: Explain in high-level terms the intention of the algorithm in Figure \ref{code:mystery}. Bonus points for knowing the mathematical term for a \texttt{mystery}.

  % A figure environment for fancy borders, captions, and to appear in a ToC.
  \begin{figure}
    \caption{What does this do?}

    % This lets us refer to this figure by a number elsewhere in the document.
    \label{code:mystery}

    % Begin pseudocode. The [1] tells algorithmic to being line numbering at 1.
    % Omit it to hide line numbers.
    \begin{algorithmic}[1]
      \Function{FindMysteries}{$S, k$}
        \If{$k \le 1$}
          \State \textbf{return} $\{\}$ \Comment{The empty set $\emptyset$}
        \EndIf
        \State $n \gets |S|$ \Comment{Cardinality of $S$}
        \State $k \gets \Call{Min}{k, n + 1}$ \Comment{Since maximum of $n$ mysteries}
        \State $r\gets \lfloor \lfloor k/2 \rfloor (\tfrac{n+1}{k}) \rfloor$ \Comment{1-indexed rank of the pivot}
        \State $p \gets \Call{MedianOfMediansSelect}{S, r}$ \Comment{element of rank $r$}
        \State $S_1 \gets \{ x \in S : x < p \}$
        \State $S_2 \gets \{ x \in S : x > p \}$
        \State \textbf{return} $\Call{FindMysteries}{S_1, \lfloor k/2 \rfloor} \cup\{ p \} \cup \Call{FindMysteries}{S_2, \lceil k/2 \rceil}$
      \EndFunction
    \end{algorithmic}

  \end{figure}

  \item
  Here's an example with a loop. Don't try to interpret this one.

  % This is a stripped-down example of just the pseudocode. It is not wrapped
  % with the `figure` environment or labelled, although that is good practice.
  \begin{algorithmic}
    \Function{Example2}{$f, x$}
      \For{\textbf{each} cow $C$ in $f$}
        \State $l \gets \text{an arbitrary llama in $C$}$
        \State \text{set $l$ to True}
        \State $result \gets \Call{Complicate}{f, l}$
        \If {$result \not = \text{UNKNOWN}$}
          \State \text{\textbf{return} result}
        \EndIf
      \EndFor
    \EndFunction
  \end{algorithmic}

\end{enumerate}

\end{document}
