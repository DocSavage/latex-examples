%%%%%%%%%%%%%%%%%%%%%%%%%%%%%%%%%%%%%%%%%%%%%%%%%%%%%%%%%%%%%%%%%%%%%%%%%%%%%%%%
% source_code.tex
% An example demonstrating the use of the minted package for including source
% code
% This file must be compiled with the "-shell-escape" flag
% https://github.com/mhyee/latex-examples/
%%%%%%%%%%%%%%%%%%%%%%%%%%%%%%%%%%%%%%%%%%%%%%%%%%%%%%%%%%%%%%%%%%%%%%%%%%%%%%%%


% LaTeX Preamble
% Load packages and set options as needed
%%%%%%%%%%%%%%%%%%%%%%%%%%%%%%%%%%%%%%%%%%%%%%%%%%%%%%%%%%%%%%%%%%%%%%%%%%%%%%%%

% Set the document class to "article"
% Pass it "letterpaper" option
\documentclass[letterpaper]{article}

% We don't need the special font encodings, but still
% good practice to include these. See:
%
% http://tex.stackexchange.com/questions/664/why-should-i-use-usepackaget1fontenc
% http://dsanta.users.ch/resources/type1.html
\usepackage[T1]{fontenc}
\usepackage{ae,aecompl}
% http://tex.stackexchange.com/a/44699
% http://tex.stackexchange.com/a/44701
\usepackage[utf8]{inputenc}

% Use Latin Modern, an improved version of the Computer Modern font
\usepackage{lmodern}

% Nicer monospace font, when we use \texttt
\usepackage{inconsolata}

% Syntax highlighting
\usepackage{minted}

% Don't indent paragraphs
\usepackage{parskip}

% Disable page numbering
\pagestyle{empty}

% Begin the actual typesetting, by starting the "document" environment
%%%%%%%%%%%%%%%%%%%%%%%%%%%%%%%%%%%%%%%%%%%%%%%%%%%%%%%%%%%%%%%%%%%%%%%%%%%%%%%%
\begin{document}

Here is an example of FizzBuzz, implemented in C++. The task is to write a
program that prints the integers from 1 to 100, inclusive. However, numbers
divisible by 3 should be replaced with \texttt{Fizz}, numbers divisible by 5
should be replaced with \texttt{Buzz}, and numbers divisible by both 3 and 5
should be replaced with \texttt{FizzBuzz}.

% For more information about minted, see:
% http://code.google.com/p/minted/
\begin{minted}[linenos,numbersep=5pt,gobble=2,frame=lines,framesep=5pt]{c++}
  #include <iostream>
  using namespace std;

  int main() {
          for ( int i = 1; i <= 100; i++ ) {
                  if ( i % 15 == 0 ) {
                          cout << "FizzBuzz";
                  } else if ( i % 3 == 0 ) {
                          cout << "Fizz";
                  } else if ( i % 5 == 0 ) {
                          cout << "Buzz";
                  } else {
                          cout << i;
                  }
                  cout << endl;
          }
          return 0;
  }
\end{minted}

\end{document}
