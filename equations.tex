%%%%%%%%%%%%%%%%%%%%%%%%%%%%%%%%%%%%%%%%%%%%%%%%%%%%%%%%%%%%%%%%%%%%%%%%%%%%%%%%
% equations.tex
% An example demonstrating the use of the amsmath package to typset equations
% https://github.com/mhyee/latex-examples/
%%%%%%%%%%%%%%%%%%%%%%%%%%%%%%%%%%%%%%%%%%%%%%%%%%%%%%%%%%%%%%%%%%%%%%%%%%%%%%%%


% LaTeX Preamble
% Load packages and set options as needed
%%%%%%%%%%%%%%%%%%%%%%%%%%%%%%%%%%%%%%%%%%%%%%%%%%%%%%%%%%%%%%%%%%%%%%%%%%%%%%%%

% Set the document class to "article"
% Pass it "letterpaper" option
\documentclass[letterpaper]{article}

% We don't need the special font encodings, but still
% good practice to include these. See:
%
% http://tex.stackexchange.com/questions/664/why-should-i-use-usepackaget1fontenc
% http://dsanta.users.ch/resources/type1.html
\usepackage[T1]{fontenc}
\usepackage{ae,aecompl}
% http://tex.stackexchange.com/a/44699
% http://tex.stackexchange.com/a/44701
\usepackage[utf8]{inputenc}

% Use Latin Modern, an improved version of the Computer Modern font
\usepackage{lmodern}

% mathtools provides support for typesetting math
% mathtools is a superset of (and fixes some bugs in) amsmath
\usepackage{mathtools}
\usepackage{amsthm}

% Not used in this example, but the following packages provide more math fonts
% and symbols
\usepackage{amsfonts}
\usepackage{amssymb}
\usepackage{mathabx}

% Don't indent paragraphs
\usepackage{parskip}

% Disable page numbering
\pagestyle{empty}


% Begin the actual typesetting, by starting the "document" environment
%%%%%%%%%%%%%%%%%%%%%%%%%%%%%%%%%%%%%%%%%%%%%%%%%%%%%%%%%%%%%%%%%%%%%%%%%%%%%%%%
\begin{document}

% For more LaTeX math examples, the Not So Short Introduction to LaTeX2e has an excellent chapter on typsetting math
% http://tobi.oetiker.ch/lshort/lshort.pdf

  % To typset math within a paragraph (text style), use $ as delimiters
  Let $[x^n]f(x)$ be defined as the \emph{coefficient extraction operator}, which extracts the coefficient of $x^n$ from the function $f(x)$. For example,
    % To break the math out of a paragraph (display style), use \begin{equation} and \end{equation}
    % This will also number the equations
    % To suppress numbering, use \begin{equation*} and \end{equation*}
    \begin{equation*}
      [x^2](23x^3-2x^2+383x-87) = -2.
    \end{equation*}
  Show that
    % \[ and \] are shorthands for \begin{equation*} and \end{equation*}
    \[ [x^3](2x+7)^5 = 3920. \]

  % The proof environment lets you end your proof with the "end of proof" symbol
  \begin{proof}
    % To align equations, use \begin{align} and \end{align}, or \begin{align*} and \end{align*}
    % Do not use eqnarray! eqnarray is old and broken, see http://www.tug.org/pracjourn/2006-4/madsen/madsen.pdf
    \begin{align*}
      % Note how "{} &" is used for alignment, and every line ends with \\
      % If you aren't breaking an equation into multiple lines, you can use "&= math" instead of "= {} & math"
      %
      % \, inserts a small space
      [x^3](2x+7)^5 = {} & [x^3] \sum_{k=0}^{5} \binom{5}{k} (2x)^{5-k}\, 7^k \\
                    = {} & [x^3] \Bigg[ \binom{5}{0} (2x)^5\, 7^0+ \binom{5}{1} (2x)^4\, 7^1 + \binom{5}{2} (2x)^3\, 7^2 \\
                         & + \binom{5}{3} (2x)^2\, 7^3 + \binom{5}{4} (2x)^1\, 7^4 + \binom{5}{5} (2x)^0\, 7^5  \Bigg] \\
                    = {} & [x^3] \binom{5}{2} (2x)^3\, 7^2 \\
                    = {} & 10 \cdot 8 \cdot 49 \\
                    % Use \qedhere to put the symbol on the same line as the last equation
                    = {} & 3920 \qedhere
    \end{align*}
  \end{proof}

\end{document}
