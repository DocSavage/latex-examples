%%%%%%%%%%%%%%%%%%%%%%%%%%%%%%%%%%%%%%%%%%%%%%%%%%%%%%%%%%%%%%%%%%%%%%%%%%%%%%%%
% entity-relationship.tex
% An example demonstrating the use of the tikz-er2 package for drawing
% entity-relationship diagrams
% https://github.com/mhyee/latex-examples/
%%%%%%%%%%%%%%%%%%%%%%%%%%%%%%%%%%%%%%%%%%%%%%%%%%%%%%%%%%%%%%%%%%%%%%%%%%%%%%%%


% LaTeX Preamble
% Load packages and set options as needed
%%%%%%%%%%%%%%%%%%%%%%%%%%%%%%%%%%%%%%%%%%%%%%%%%%%%%%%%%%%%%%%%%%%%%%%%%%%%%%%%

% Set the document class to "article"
% Pass it "letterpaper" option
\documentclass[letterpaper]{article}

% We don't need the special font encodings, but still
% good practice to include these. See:
%
% http://tex.stackexchange.com/questions/664/why-should-i-use-usepackaget1fontenc
% http://dsanta.users.ch/resources/type1.html
\usepackage[T1]{fontenc}
\usepackage{ae,aecompl}

% Entity-relationsip package
\usepackage{tikz-er2}
% Include tikz library for more control over positioning
\usetikzlibrary{positioning}
% Styling for entities, attributes, and relationships
\tikzstyle{every entity} = [draw=blue!50!black!100, fill=blue!20]
\tikzstyle{every attribute} = [draw=yellow!50!black!100, fill=yellow!20]
\tikzstyle{every relationship} = [draw=green!50!black!100, fill=green!20]

% Don't indent paragraphs
\usepackage{parskip}

% Disable page numbering
\pagestyle{empty}

% Begin the actual typesetting, by starting the "document" environment
%%%%%%%%%%%%%%%%%%%%%%%%%%%%%%%%%%%%%%%%%%%%%%%%%%%%%%%%%%%%%%%%%%%%%%%%%%%%%%%%
\begin{document}

% This is a very, very basic example of how to draw an ER diagram with Tikz
% The package also allows for weak entities, aggregations, and inheritance.
% Positioning, edge labels, and edge styles are done the normal Tikz way,
% the package just provides shortcuts for the common shapes and relationships
% from the ER model.

% You can find the package and documentation at:
% https://www.assembla.com/wiki/show/tikz-er2
\begin{center}
\begin{tikzpicture}[node distance=6em, every edge/.style={link}]

  \node[entity] (emp) {Employee};
    \node[attribute] (e_id) [left=2em of emp] {\key{ID}} edge (emp);
    \node[attribute] (e_name) [right=2em of emp] {Name} edge (emp);
    \node[attribute] (e_sal) [above=2em of emp] {Salary} edge (emp);

  \node[relationship] (works) [below of=emp, node distance=8em] {WorksIn} edge (emp);

  \node[entity] (dept) [below of=works, node distance=8em] {Department} edge (works);
    \node[attribute] (d_num) [left=2em of dept] {\key{Num}} edge (dept);
    \node[attribute] (d_name) [right=2em of dept] {Name} edge (dept);

\end{tikzpicture}
\end{center}

\end{document}
